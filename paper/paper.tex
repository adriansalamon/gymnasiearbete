\usepackage[utf8]{inputenc}
\usepackage[english]{babel}
\usepackage[table,xcdraw]{xcolor}
\usepackage{hyperref}
\usepackage{float}
\usepackage{amsmath}
\usepackage{mathtools}
\usepackage{tikz}
\usepackage{pgfplots}
\usepackage{blindtext}

\pgfplotsset{compat=1.14} % Disable-lint

\hypersetup{%
	colorlinks,
	citecolor=black,
	filecolor=black,
	linkcolor=black,
	urlcolor=black
}

\usepackage[backend=biber,style=authoryear]{biblatex}
\usepackage[newfloat]{minted}
\usemintedstyle{vs}
\usepackage{csquotes}
\usepackage{caption}
\usepackage{booktabs}


\graphicspath{ {images/} }

\newcommand\tab[1][1cm]{\hspace*{#1}}
\SetupFloatingEnvironment{listing}{name=Code example}
\definecolor{bg}{rgb}{0.95,0.95,0.95}
\newenvironment{code}{\captionsetup{type=listing}}{}

\newcommand\codeblock[3]{%
\begin{code}
	\caption{#3}
	\inputminted[%
	    mathescape,
	    linenos,
	    numbersep=5pt,
	    tabsize=4,
	    label=Helloworld,
	    bgcolor=bg,
	    breaklines%
	]{TypeScript}{#1}
	\label{#2}
\end{code}
}

\newcommand\Code[1]{\texttt{#1}}

\usepackage{hyperref}
\usepackage{float}
\usepackage{amsmath}
\usepackage{mathtools}

\newcommand\tab[1][1cm]{\hspace*{#1}}


\hypersetup{
	colorlinks,
	citecolor=black,
	filecolor=black,
	linkcolor=black,
	urlcolor=black
}
\usepackage{pseudocode}
\usepackage{tikz}
\usepackage{pgfplots}
\pgfplotsset{compat=1.14}


\addbibresource{sample.bib}


\begin{document}

\begin{titlepage}
	\centering
	\vspace{2cm}
	{\Huge Analysis of voting systems: What should I call it? \par}
	\vspace{0.6cm}
	{\LARGE Adrian Salamon\par}

	\vspace{0.6cm}
	{\Large Kungsholmens gymnasium\par}
	\vspace{0.4cm}
	{\large Senior thesis\par}
	\vspace{0.6cm}
	\includegraphics[width=0.3\textwidth]{kg}\par\vspace{1cm}
	\vspace{4cm}
	\vfill
	Supervised by: \par
	Maja Kankaanranta

	\vfill

	% Bottom of the page
	{\large \today\par}
\end{titlepage}

\pagebreak

\begin{abstract}
	The abstract text goes here.
\end{abstract}

\pagebreak

\tableofcontents

\pagebreak

\section{Introduction}
\subsection{Background}
Taking a collective decision as a population is difficult. To solve this issue, voting systems with defined rules are used. They are used to show common preferences within a population, for example what politician a population wants to see elected. Several types of systems have been designed and there are a myriad of variations of those systems. They range from simple methods such as “most votes win” to complex processes that can only be practically carried out by a computer. However, practically all voting systems are algorithmic in nature, which makes them interesting to study for both Computer Scientists and Mathematicians. When deciding on what systems that shall be used, it is useful to know what differences and similarities they have.
\subsection{Purpouse}
The purpose of this essay is to evaluate differences and similarities in terms of election results in three different voting systems: Single Transferable Vote (STV), First-Past-The-Post (FPTP) and the Schulze Method. I will only be investigating and discussing differences and similarities in the results of the methods, not touching on how practically feasible the systems are in an actual election.
\pagebreak
\section{Theory}
Theory will be here some day
\section{Methodology}
Methodology here
All voting methods have been implemented in Typescript – mostly due to the author having previous experience with JavaScript. Typescript is a superset of JavaScript and compiles down to it. Typescript has some extra features compared to JavaScript but most importantly it has optional static type checking. Here is some example code in typescript:
\begin{lstlisting}
let helloWorld: string = 'Hello World' // This is a comment
console.log(helloWorld)
\end{lstlisting}
\subsection{Modeling test data}
\pagebreak
\section{Specification of voting methods}
This paper compares the results of theese voting methods
\subsection{First-past-the-post}
\subsubsection{Description}
The first-past-the-post voting method is a method desiged for electing one or several candidates out of a set of candidates. Each voter has one (1) vote which may be allocated to any candidate. In a first-past-the-post election the N candidates with most votes get elected.
\subsubsection{Justification}
The method is simple and can easily be understood. The first-past-the-post method is widely used in for example the United Kingdom and United States. Comparisons with this method can also be easily made.
\subsubsection{Pseudocode}
let $V$ be the set of votes with $V_{i}$ being the number of votes for candidate $i$ \\
let $N$ be number of candidates to be elected \\
sort $V$ based on $V_{i}$\\
slice $V$ between $0$ and $N$
\subsubsection{Implementation}
The first-past-the-post program is implemented in a single file. It takes the following inputs:
\begin{lstlisting}
ballots: number[][] // Ex. [[0,1,2],[1,0,2],...]
seats: number // Ex. 2
\end{lstlisting}
The ballot data is first mapped to only include first prefrences. Then the algorithm loops over the ballots and gets the sum of votes for each candidate:
\begin{lstlisting}
let results: Result[] = []
// Loops over prefrences and adds/increments result
for (var i = 0; i < firstPrefrences.length; i++) {
    // Finds index in results array of candidate at i
    let index = results.findIndex(obj => obj.cand === firstPrefrences[i])
    // If candidate is not added to results
    if (index === -1) {
        results.push({ cand: firstPrefrences[i], votes: 1 })
    } else {
        // Increment count if already added in results
        results[index].votes += 1;
    }
}
\end{lstlisting}
The results array is sorted and sliced to only include the $N$ candidates with most votes:
\begin{lstlisting}
// Sorts array by number of votes
results.sort((a, b) => {
    if (a.votes > b.votes) {
        return -1
    } else if (a.votes < b.votes) {
        return 1
    }
    return 0
})
// Gets the first elements in the sorted array i.e the winners
let winners = results.slice(0, seats)
// Returns array of winners
return winners.map(cand => cand.cand)
\end{lstlisting}
\subsection{Single transferable vote}
\subsubsection{Description}
The Single Transferable Vote method is a more complicated algorithm than the first-past-the-post method. Voters are given ballots where they are supposed to rank candidates in order. The specific rules for if you need to rank all candidates or rank several candidates at the same value not can vary. For the sake of simplicity, in this implementation, all candidates must be ranked at unique and linear values. A ballot may look like this:
\begin{figure}[H]
	\centering
	\includegraphics[height=140px]{ballot}
	\caption{Sample STV ballot}
	\label{STV ballot}
\end{figure}
This means that there is more data availible for detrermining the result of the election which the method can take advantage of. Votes are, if needed, transfered between choices within a ballot, hence the name Single Transferable Vote. The method relies on the concept of a quota, the number of votes you need to be elected. The Droop-Quota, defined as $(\frac{\text{total valid poll}}{\text{seats} + 1})+1$, is often used. If a candidate recieves equal or more votes than the quota, he/she is elected. If the number of votes exxeed the quota, a fraction of the votes are transfered to the next choice on the ballots that voted for the elected candidate. This process is repeated until there are no candidates with more votes than the quota. If there still are seats yet to be filled, the candidate with fewest votes is eliminated and his/her votes are transfered to the next choice on those ballots. This process is repeated until all seats are filled. The process can become quite complex with many ballots and transfering fractional votes. In order to resolve large scale elections, a computer in essential.
\subsubsection{Justification}
STV is an established voting method in use in several countries. It is used for parlimentary elections in Ireland, Malta and Australia as well as being used in local and regional elections across the world.
\subsubsection{Pseudocode}
\label{Stv psuedocode}
let $quota$ be the quota \\
let $seats$ be the number of seats \\
let $winners$ be the list of elected candidates \\
let $votes$ be the set of votes with $votes_{i}$ signinfying votes for candidate $i$\\
while $\left\vert{winners}\right\vert < seats$\\
\tab if any candidate $i \notin winners$ and $votes_{i} \geq quota$ \\
\tab\tab add $i$ to $winners$ \\
\tab\tab continue \\
\tab end if \\
\tab if any candidate $i$ where $votes_{i} > quota$\\
\tab\tab multiply $votes_{i}$ with $(1 - \frac{\text{surplus votes}_{i}}{votes_{i}})$ \\
\tab\tab multiply next prefrencences for $votes_{i}$ with $\frac{\text{surplus votes}_{i}}{votes_{i}}$ and \\ \tab\tab distribute into $votes$\\
\tab\tab continue \\
\tab end if \\
\tab if $\left\vert{votes}\right\vert = seats$ \\
\tab\tab add all $votes \notin winners$ to $winners$ \\
\tab\tab continue \\
\tab end if \\
\tab $k \coloneqq$ index of smallest $votes_{i}$ \\
\tab eliminate $votes_{k}$ and distribute all next-prefrence votes for candidate $k$ \tab into $votes$\\
end while
\subsubsection{Implementation}
The single transferable vote algorithm uses a tree structure to represent the state of the election. Every candidate is a top-level node with an attribute showing the number of current votes for the candidate. Each node has child nodes that show the next prefrences of the voters for that particular candidate, see figure \ref{Tree structure}. It is easy to manipulate branches via recursion. Merging and multiplying branches is used when transfering votes from one candidate node to another. A candidate node is a Typescript class. It can be found in the \texttt{stv/candidate.ts} file and has the following properties and functions:
\lstinputlisting[language=JavaScript]{code/candidateNode.ts}
\begin{figure}[H]
	\centering
	\includegraphics[height=200px]{Tree}
	\caption{Simplified and shortened version of the tree structure. Each candidate is a top-level node with children, signifying the prefrences of the voters for that candidate.}
	\label{Tree structure}
\end{figure}
With those functions, we can recursively manipulate branches and trees. In order to distribute the child nodes of an eliminated candidate, we just need to add each of those nodes to the corresponding top level nodes. In order to transfer surplus votes, we can multiply each child node with a factor and then distribute those nodes onto the top level nodes. The full source code for the tree manipuation functions can be found in \texttt{stv/tree.ts}.

The input into the algorithm is a set of ballots. A ballot consists of an ordered list of candidates according to prefrence. Every candidate is represented by an integer. The input data is converted into the tree structure via the \texttt{buildTree} function. It can also be found in the \texttt{stv/tree.ts} file.

The main procces of the program is found in the \texttt{stv/election.ts} file and runs the process defined in the psuedocode on page \pageref{Stv psuedocode}. The program takes the input defined above and a number representing the number of seats to elect as inputs. The process outputs an array of winning candidates as well as a log file.

\subsection{Schulze}
\subsubsection{Description}
The Schulze Method is a method used mainly to determine results in a single-winner election but can also be used to provide rankings and elect multiple candidates in a single election. Ballots are identical to those of the single transferable vote, see figure \ref{STV ballot} on page \pageref{STV ballot}. The method is compliant with the Condorcet Criterion, which means that if there is a candidate who the majority prefers in a pairwise comparison with every other candidate, that candidate wins. As mentioned, this method relies on comparing the prefrences of every candidate to one another as shown in table \ref{Pairwise comparison matrix}. If there is a condorcet winner the process is simple: declare the condorcet winner a winner and run another iteration without that candidate. Repeat this process until all seats are filled. However, there can be condorcet ties, where there is no condorcet winner. There are multiple ways of resolving the tie, one of them being the Schulze Method.

The Schulze Method resolves ties by investigating possible paths between candidates. For example, if out of a total of 50 ballots 30 ballots prefer $B>A$, 28 ballots prefer $A>C$ and 27 ballots prefer $C>B$, a path from $A$ to $B$ can be created by going $A \rightarrow C \rightarrow B$. The path is said to have the $strength$ of the weakest link in the path. In this example, the path strenght between A and B is 27 due to the link between $C$ and $B$ having a strenght of 27. There may be several paths between two candidates and the goal of the process is to find the strongest path between any candidate $A$ and $B$, written as $p[A,B]$. By finding the strongest path between all nodes a result can be obtained where $p[X,Y] \geq p[Y,X]$ wich means that candidate $X$ wins. The Shulze Method also provides a linear ranking between all candidates, for example that $E > B > A > C > D$. By selecting the $N$ top candidates you can use the method for a multi-seat election. As the process can become very complex as the number of candidates grow, a computer is needed to resolve large elections.

The difficult problem in this method is finding the strongest path between candidates. The problem is called the widest path problem in graph theory. An efficent and relatively simple way to compute this problem is via the Floyd-Warshall algorithm which is used in the implementation. See \ref{Schulze psuedocode} for the full algorithm.

\begin{table}[H]
\centering
\begin{tabular}{l|c|c|c|c|c|}
\cline{2-6}
 & \multicolumn{1}{l|}{A} & \multicolumn{1}{l|}{B} & \multicolumn{1}{l|}{C} & \multicolumn{1}{l|}{D} & \multicolumn{1}{l|}{E} \\ \hline
\multicolumn{1}{|l|}{A} & \cellcolor[HTML]{9B9B9B} & \cellcolor[HTML]{FFDDDD}17 & \cellcolor[HTML]{FFDDDD}14 & \cellcolor[HTML]{DDFFDD}35 & \cellcolor[HTML]{DDFFDD}30 \\ \hline
\multicolumn{1}{|l|}{B} & \cellcolor[HTML]{DDFFDD}33 & \cellcolor[HTML]{9B9B9B} & \cellcolor[HTML]{FFDDDD}24 & \cellcolor[HTML]{DDFFDD}47 & \cellcolor[HTML]{DDFFDD}36 \\ \hline
\multicolumn{1}{|l|}{C} & \cellcolor[HTML]{DDFFDD}36 & \cellcolor[HTML]{DDFFDD}26 & \cellcolor[HTML]{9B9B9B} & \cellcolor[HTML]{DDFFDD}40 & \cellcolor[HTML]{DDFFDD}42 \\ \hline
\multicolumn{1}{|l|}{D} & \cellcolor[HTML]{FFDDDD}15 & \cellcolor[HTML]{FFDDDD}3 & \cellcolor[HTML]{FFDDDD}10 & \cellcolor[HTML]{9B9B9B} & \cellcolor[HTML]{FFDDDD}18 \\ \hline
\multicolumn{1}{|l|}{E} & \cellcolor[HTML]{FFDDDD}20 & \cellcolor[HTML]{FFDDDD}14 & \cellcolor[HTML]{FFDDDD}8 & \cellcolor[HTML]{DDFFDD}32 & \cellcolor[HTML]{9B9B9B} \\ \hline
\end{tabular}
\caption{Pairwise comparison matrix used in the Schulze Mehtod. For example, 17 ballots prefer $A>B$ and 33 ballots prefer $B>A$.}
\label{Pairwise comparison matrix}
\end{table}
\subsubsection{Justification}
The Shulze Method is used by multiple organisations around the world. It has been used by various software organisations such as The Wikimedia Foundation, The Debian Project and Ubuntu. It is also used by political parties such as the Pirate Party in Sweden and various other countries. The process also differs greatly in both method and implementation from the two other voting methods used in this paper.
\subsubsection{Pseudocode}
\label{Schulze psuedocode}
let $d[i,j]$ be the number of ballots that prefer candidate $i$ to candidate $j$\\
let $p[i,j]$ be the strength of the strongest path from candidate $i$ to candidate $j$\\
for $i$ from 1 to $C$\\
\tab for $j$ form 1 to $C$\\
\tab\tab if ($i \ne j$)\\
\tab\tab\tab if ($d[i,j] > d[j,i]$)\\
\tab\tab\tab\tab $p[i,j] \coloneqq d[i,j]$\\
\tab\tab\tab else\\
\tab\tab\tab\tab $p[i,j] \coloneqq 0$\\
\tab\tab\tab endif \\
\tab\tab endif \\
\tab endfor \\
endfor\\
for $i$ from 1 to $C$\\
\tab for $j$ form 1 to $C$\\
\tab\tab if ($i \ne j$)\\
\tab\tab\tab for $k$ from 1 to $C$\\
\tab\tab\tab\tab if ($i \ne k \text{ and } j \ne k$)\\
\tab\tab\tab\tab\tab $p[j,k] \coloneqq \text{max }(p[j,k], \text{ min }(p[j,k], p[i,k]))$\\
\tab\tab\tab\tab endif \\
\tab\tab\tab endfor \\
\tab\tab endif \\
\tab endfor \\
endfor\\

\subsubsection{Implementation}

\section{Evaluation of methods}
\subsection{Sample I}
\subsection{Sample II}
\subsection{Sample III}

\section{Conclusion}





%\subsection{ Heading Here}
%This is some code.
%\lstinputlisting[language=JavaScript]{code/examples/example.js}



%\section{Conclusion}
%Write your conclusion here. And here is a citation:
%\textcite{sigfridsson2}

%And another one here (\textcite{sigfridsson})

%\section{Source code}
%\lstinputlisting[language=JavaScript]{code/test.js}

\pagebreak




\begin{figure}
	\centering
	\begin{tikzpicture}
		\begin{axis}[
			ybar stacked,
			width = 0.9\textwidth,
			height = 0.5\textwidth,
			cycle list = {black!80,black!50,black!20},
			legend style={at={(0.5,-0.25)},
			anchor=north,legend columns=-1},
      x tick style = transparent,
			ylabel = {Elections won},
			xlabel = {Party},
			ymajorgrids = true,
			every axis plot/.append style={fill,draw=none,no markers},
      ymin = 0
      ]
			\addplot coordinates
			{(0, 1)(1, 1)(2, 1)(3, 0)(4, 0)(5, 0)(6, 0)};
			\addplot coordinates
			{(0, 1)(1, 1)(2, 1)(3, 0)(4, 0)(5, 0)(6, 0)};
			\addplot coordinates
			{(0, 0)(1, 1)(2, 1)(3, 0)(4, 1)(5, 0)(6, 0)};
			\legend{FPTP, STV, Schulze}
		\end{axis}
	\end{tikzpicture}
\end{figure}

\begin{figure}
	\centering
	\begin{tikzpicture}
		\begin{axis}[
			ybar,
			width = 0.9\textwidth,
			height = 0.5\textwidth,
			legend style={at={(0.5,-0.25)},
			anchor=north,legend columns=-1},
      x tick style = transparent,
			ylabel = {First prefrence votes},
			xlabel = {Party},
			ymajorgrids = true,
			every axis plot/.append style={fill,draw=none,no markers},
      ymin = 0
      ]
			\addplot coordinates
			{(0, 17)(1, 36)(2, 32)(4, 9)(5, 5)(6, 1)};
		\end{axis}
	\end{tikzpicture}
\end{figure}



\printbibliography


\section{Appendix}
First-past-the-post program


\end{document}
